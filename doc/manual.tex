\documentclass{article}
\usepackage{amsmath}

\newcommand{\BigO}[1]{\ensuremath{\operatorname{O}\bigl(#1\bigr)}}
\begin{document}
	\begin{center}
	{\Huge \bf Sudoku $M*N$ Solver} \\
	{\large Daniel Lovasko} \\[1\baselineskip]
	\end{center}
\section*{Introduction}
Sudoku solver written entirely in Prolog. Supports only square-shaped puzzles,
but inner areas are allowed to be rectangular (ex. $20*20$ puzzle may have inner
areas $5*4$, $4*5$, $2*10$, $10*2$ configured in inverse). GNU Prolog was chosen as
Prolog flavour, mainly due to licensing terms and nice IO library.

\section{Goal} The goal of this solver is to find all possible solutions for
the supplied puzzle using chosen solving method.  

\section{Command Line Arguments} The goal of this solver is to find all
possible solutions for the supplied Multiple argument-counts are acceptable:
\paragraph{0 arguments} user will be prompted after startup to supply path to
the file containing the sudoku puzzle and solving method name (for solving
method names and information, see Algorithms section) 
\paragraph{1 argument} the first argument will be understood as path to the
file containing the sudoku puzzle, user will be prompted for solving method name 
\paragraph{2 arguments} first argument will be understood as path to the file
containing the sudoku puzzle, second will be understood as solving method name
\\
\\
NOTE: All supplied information during runtime (not command line arguments) must
be surrounded with "".
\section{Data Format}
Puzzle is loaded from external file saved on the disk. This file must be in the
following format: First line consists of two integers, each followed by a dot.
These numbers represent width and height of inner cell of the puzzle,
respectively.
Next, the actual puzzle is expected. Each cell is represented as integer,
ranging from 1 to width*height followed by a dot. 0, followed also by a dot, is
also possible, meaning this cell is not solved yet and is a subject to our
solving algorithm. Number of cells is always (width*height) * (width*height),
so for example: width = 3 and height = 3, 81 cells are expected. Be careful to
provide exact number of cells, as program can behave strangely with incorrect
input.
\subsection{Example Puzzle File}
\begin{verbatim}
2. 2.
1. 0. 2. 0.
2. 0. 0. 0.
0. 3. 0. 0.
0. 0. 0. 4.
\end{verbatim}
\section{Data Structures and API}
The puzzle is represented by sudoku/3 predicate. First argument is M, the width
of each inner area, second N, the height of each inner area and finally Fields,
the list of fields. Fields are the inner program representation of the
user-supplied cells loaded from the puzzle file. Each field is represented by
list of length 3, first two elements being the X, Y coordinates, last element
being Value of that field.
Manipulation with such data structure can be found in get set.pl source file.
Read-only access to specified row can be done via filter same row/3, with
analogue procedures filter same column/3 and filter same area/3 for column and
area respectively. These procedures take O􏰈(M ∗ N)2􏰉 time.
Write access is supported with set value/4 procedure. This procedure takes O􏰈(M
∗ N)2􏰉 time.
\section{Algorithms}
There are two basic approaches used in this program. First, the static solving
algorithm, which creates a list of fields at first, and than backtracks
thought this list, always inserting correct values to the puzzle. During the
first computation, no actual data modification is made and the heuristic method
does not take the values to account.
On the other hand, dynamic solving algorithm modifies actual puzzle data during its
run. The heuristic method is based on field values.
Common code shared between these algorithms include sudoku puzzle get/set API
calls and predicates concerning field correctness.
\subsection{Static Algorithms}
The algorithm can be divided into two steps. First step is creating list of
fields that were not pre-solved. Order of fields in this list based on
heuristic approach.
Each individual technique is described in its subsection. Second step is using
Prolog’s internal back-tracking engine to try to fill in each field, element by
element.
\subsubsection{Method 1: Relatives}
This heuristic approach sorts fields by following criterion: more fields in
same column, row or area already solved or added to list. These all are called
the ’relatives’. In each step, it chooses the one with the most ’relatives’.
\subsubsection{Method 2: Neighbours}
This heuristic approach sorts fields by following criterion: more fields solved
or added to list from adjacent fields. These all are called the ’neighbours’.
In each step, it chooses the one with the most ’neighbours’.
\subsection{Method 3: Adhoc (Dynamic Algorithm)}
In each step of this algorithm, all yet unsolved fields are evaluated by
heuristic - fields with less possible correct values are ranked higher - and
than filled in. No list is being created, the actual puzzle is modified, so
each step needs to recalculate the heuristic values.
\section{Generator}
IMPORTANT: I myself did not write this code, just copied it from FreeBSD ports
collection. My contribution is the interface described below.
\subsection{make}
This command handles the compilation of the C source code. libcurses is needed
for compilation.
\subsection{make generate}
Uses the compiled binary sudoku generator to create a set of puzzles with size
3x3.
\subsection{make solve}
Invokes the compiled binary sudoku solver (written in Prolog) with each
generated puzzle, for each solving technique.

\end{document}
